\chapter*{Введение}
\addcontentsline{toc}{chapter}{Введение}

Компьютерная графика является одной из неотъемлемых частей современной жизни. Она используется повсеместно: в кинофильмах и мультфильмах, в компьютерных играх, а также для наглядного отображения данных.
В связи с этим возрастает потребность в инструментах, позволяющих создавать трехмерные модели различной сложности. На данный момент существует множество подобных инструментов, предоставляющих обширный спектр различных функций, упрощающих процесс создания 3d модели. 

\textbf{Цель данного курсового проекта} --- разработка редактора трехмерных моделей, предоставляющего пользователю возможность производить базовые преобразования над полигональной моделью и ее составляющими частями (вершинами, ребрами, гранями).

Для достижения поставленной цели необходимо решить следующие задачи:
\begin{itemize}[label=---]
	\item изучить и провести анализ существующих алгоритмов компьютерной графики, выбрать наиболее подходящие из них для реализации редактора трехмерных моделей;
	\item спроектировать программное обеспечение, предоставляющее пользователю необходимые функции;
	\item реализовать спроектированное программное обеспечение;
	\item провести сравнительный анализ последовательной и параллельной реализаций алгоритма Z-буфера.
\end{itemize}