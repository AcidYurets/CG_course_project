\chapter{Исследовательский раздел}

В данном разделе приведён пример работы программы, а также проведён сравнительный анализ многопоточной и однопоточной реализаций.

\section{Технические характеристики}

Замеры времени выполнялись на личном ноутбуке. Технические характеристики устройства, на котором выполнялось тестирование представлены далее:

\begin{itemize}[label=---]
	\item операционная система Windows 10 Домашняя;
	\item память 16 Гбайт;
	\item процессор 3.20 ГГц 4‑ядерный процессор Intel Core i5 11-го поколения;
	\item процессор имеет 4 физических и 8 логических ядер.
\end{itemize}

Во время замеров ноутбук был включен в сеть электропитания.

Для корректности сравнения многопоточной и однопоточной реализаций была выключена оптимизация при компиляции.

\section{Время выполнения алгоритмов}

Для замера времени работы алгоритмов использовалась функция \linebreak 
chrono::high\_resolution\_clock::now(...) из библиотеки chrono \cite{chrono} на C++.

Результаты замеров времени работы многопоточной и однопоточной реализаций алгоритмов приведены в таблице \ref{tbl:time}. Строчка с количеством потоков 0 означает не распараллеленную реализацию алгоритма. На рисунке \ref{fig:time} приведена графическая интерпретация замеров времени. Замеры производились по 50 раз для каждого числа порожденных потоков на модели, которая представляет собой квадратную плоскость, состоящую из двух граней.  

\begin{table}[h]
	\begin{center}
		\begin{threeparttable}
			\captionsetup{justification=raggedright,singlelinecheck=off}
			\caption{Результаты замеров времени}
			\label{tbl:time}
			\begin{tabular}{|c|c|}
				\hline
				Количество порожденных потоков & Время, мкс \\ \hline
				        0          &   254569   \\ \hline
				        1          &   293457   \\ \hline
				        2          &   207791   \\ \hline
				        4          &   128807   \\ \hline
				        8          &   84151    \\ \hline
				        16         &   66569    \\ \hline
				        32         &   61264    \\ \hline
				        64         &   65488    \\ \hline
				       128         &   75354    \\ \hline
			\end{tabular}
		\end{threeparttable}
	\end{center}
\end{table}


\begin{figure}[h!]
	\begin{center}
		\begin{tikzpicture}
			\begin{axis}[
				legend pos = north east,
				xlabel=Количество порожденных потоков,
				ylabel=Время в миксосекундах,
				minor tick num = 1,
				grid = both,
				major grid style = {lightgray},
				minor grid style = {lightgray!25},
				%xtick distance = 50,
				width = 0.9\textwidth,
				height = 0.7\textwidth]
				
				\addplot[
				blue,
				semithick,
				mark = x,
				mark size = 3pt,
				thick,
				] file {inc/data/time_evaluation_res.tsv};
				
				\legend{Алгоритм обработки грани}
			\end{axis}
		\end{tikzpicture}
	\end{center}
	\caption{Время работы реализации алгоритма обработки грани в зависимости от количества потоков}
	\label{fig:time}
\end{figure}
\clearpage


Из результатов замеров видно, что при числе порожденных потоков M~$=$~1, многопоточный алгоритм работает в 1.15 раз дольше, чем алгоритм с отсутствующей параллельностью. Это происходит из-за затрат на диспетчеризацию этого потока.

Тем не менее, при M~$=$~2 и более, многопоточный алгоритм начинает сильно выигрывать по времени у не распараллеленного алгоритма. Наибольшую эффективность по времени работы удается достичь при M~$=$~32, тогда многопоточный алгоритм работает примерно в 4.1 раза быстрее, чем алгоритм без использования многопоточности. Это происходит из-за того, что при M~$>$~32 затраты на диспетчеризацию M потоков превышают преимущество от использования многопоточности.

Также важно заметить, что замеры времени производились на программе, скомпилированной без оптимизаций по скорости работы. При включенной оптимизации результаты могут отличаться.

\section*{Вывод}
Многопоточная реализация с использованием 32 порожденных потоков показала наилучший результат. Такая реализация оказалась эффективнее примерно в 4.1 раз чем реализация, не использующая многопоточность.

Рекомендуется использовать число порожденных потоков примерно в 4 раза большее, чем число логических ядер.