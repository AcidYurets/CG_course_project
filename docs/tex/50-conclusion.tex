\chapter*{Заключение}
\addcontentsline{toc}{chapter}{Заключение}

\textbf{Поставленная цель была достигнута} --- разработан редактор трехмерных моделей, предоставляющий пользователю возможность производить базовые преобразования над полигональной моделью и ее составляющими частями (вершинами, ребрами, гранями).

Для достижения поставленной цели были решены следующие задачи:
\begin{itemize}[label=---]
	\item изучены существующие алгоритмы компьютерной графики и проведен их анализ. Выбраны наиболее подходящие из них для реализации редактора трехмерных моделей;
	\item спроектировано программное обеспечение, предоставляющее пользователю необходимые функции;
	\item реализовано спроектированное программное обеспечение;
	\item проведен сравнительный анализ последовательной и параллельной реализаций алгоритма Z-буфера.
\end{itemize}

Разработанный программный продукт в режиме реального времени синтезирует трехмерное изображение при помощи алгоритмов компьютерной графики. Программа реализована таким образом, что пользователь может добавлять новые объекты на сцену, производить трансформацию над их составными частями, изменять положение источника света. 

Из результатов проведенного сравнительного анализа следует, что наилучший результат показала многопоточная реализация с использованием 32 порожденных потоков. Таким образом, наиболее оптимальное число порожденных потоков примерно в 4 раза превышает число логических ядер.

